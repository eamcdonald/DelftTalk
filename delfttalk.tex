\documentclass{beamer}



%\usepackage{beamerthemesplit}
\usetheme{Boadilla}
%\usetheme{default}
%\useinnertheme{rounded}

%\useoutertheme{shadow}
\usecolortheme{rose}
%\usefonttheme{serif}
\setbeamertemplate{navigation symbols}{}
\usetheme{Madrid}

\usepackage{amssymb,amsmath,amscd,amsfonts,amsthm,dsfont,color,graphicx}
\usepackage{amscd}
%\usepackage[numbers]{natbib}
% \usepackage[french]{babel}
%\usepackage[active]{srcltx}


\def\qd{\,{\mathchar'26\mkern-12mu d}}

% \date[]{}

 \newcommand\makebeamertitle{\frame{\maketitle}}%

 \AtBeginDocument{
   \let\origtableofcontents=\tableofcontents
   \def\tableofcontents{\@ifnextchar[{\origtableofcontents}{\gobbletableofcontents}}
   \def\gobbletableofcontents#1{\origtableofcontents}
 }
\numberwithin{equation}{section}
  \theoremstyle{plain}
  \newtheorem*{thm*}{\protect\theoremname}
  \theoremstyle{plain}
  \newtheorem*{cor*}{\protect\corollaryname}
 \theoremstyle{definition}
 \newtheorem*{defn*}{\protect\definitionname}
 \theoremstyle{plain}
\newtheorem*{lem*}{\protect\lemmaname}
  \theoremstyle{plain}
  \newtheorem*{rem*}{\protect\remarkname}
   \theoremstyle{definition}
 \newtheorem*{prop*}{\protect\propositionname}

\usetheme{Madrid}

\makeatother

  \providecommand{\corollaryname}{Corollary}
  \providecommand{\definitionname}{Definitioninition}
  \providecommand{\theoremname}{Theorem}
   \providecommand{\lemmaname}{Lemma}
   \providecommand{\remarkname}{Remark}
   \providecommand{\propositionname}{Proposition}
   
   
\newcommand{\Rl}{\mathbb{R}}
\newcommand{\Cplx}{\mathbb{C}}
\newcommand{\Itgr}{\mathbb{Z}}
\newcommand{\Ntrl}{\mathbb{N}}
\newcommand{\Circ}{\mathbb{T}}
\newcommand{\Sb}{\mathbb{S}}
\newcommand{\Disc}{\mathbb{D}}
\newcommand{\Aff}{\mathbb{A}}

% The Caligraphic alphabet
\newcommand{\Ac}{\mathcal{A}}
\newcommand{\Bc}{\mathcal{B}}
\newcommand{\Cc}{\mathcal{C}}
\newcommand{\Dc}{\mathcal{D}}
\newcommand{\Ec}{\mathcal{E}}
\newcommand{\Fc}{\mathcal{F}}
\newcommand{\Gc}{\mathcal{G}}
\newcommand{\Hc}{\mathcal{H}}
\newcommand{\Ic}{\mathcal{I}}
\newcommand{\Jc}{\mathcal{J}}
\newcommand{\Kc}{\mathcal{K}}
\newcommand{\Lc}{\mathcal{L}}
\newcommand{\Mv}{\mathcal{M}}
\newcommand{\Nv}{\mathcal{N}}
\newcommand{\Oc}{\mathcal{O}}
\newcommand{\Pc}{\mathcal{P}}
\newcommand{\Qc}{\mathcal{Q}}
\newcommand{\Rc}{\mathcal{R}}
\newcommand{\Sc}{\mathcal{S}}
\newcommand{\Tc}{\mathcal{T}}
\newcommand{\Uc}{\mathcal{U}}
\newcommand{\Vc}{\mathcal{V}}
\newcommand{\Wc}{\mathcal{W}}
\newcommand{\Xc}{\mathcal{X}}
\newcommand{\Yc}{\mathcal{Y}}
\newcommand{\Zc}{\mathcal{Z}}


\newcommand{\Sp}{\mathrm{Sp}}
\newcommand{\tr}{\mathrm{tr}}
\newcommand{\Op}{\mathrm{Op}}
\newcommand{\sym}{\mathrm{sym}}
\newcommand{\Vol}{\mathrm{Vol}}
\newcommand{\Tr}{\mathrm{Tr}}
\newcommand{\dist}{\mathrm{dist}}
\newcommand{\sgn}{\operatorname{sgn}}
\newcommand{\diag}{\mathrm{diag}}
\newcommand{\id}{\mathrm{id}}
\newcommand{\Poly}{\mathrm{Poly}}

\newcommand{\spec}{\mathrm{Spec}}
\newcommand{\abs}{\mathrm{abs}}

\newcommand{\CV}{\mathrm{CV}}
\newcommand{\PCV}{\mathrm{PCV}}


% Used for highlighting. To remove all highlighting just make the command blank
\newcommand{\hl}{\color{red}}



\newcommand{\dom}{\mathrm{dom}}
\newcommand{\Bl}{\mathbb{B}^4}
\newcommand{\supp}{\mathrm{supp}}
\newcommand{\BS}{\mathfrak{BS}}
\newcommand{\dyad}{\mathrm{dyad}}
\newcommand{\Qs}{\mathscr{Q}}
\newcommand{\Av}{\mathrm{Av}}
\newcommand{\loc}{\mathrm{loc}}
% DOI transformer
\newcommand{\Ti}{\mathcal{T}}
\newcommand{\sa}{\mathrm{sa}}


\newcommand{\Str}{\operatorname{Str}}
   
\begin{document}

\title[Lipschitz estimates for $p<1$]{Lipschitz estimates in quasi-Banach Schatten ideals}


\author[E. McDonald]{Ed McDonald\\
Joint with F. Sukochev.}


\institute[]{Penn State University}

\makebeamertitle


\begin{frame}{Introduction}
This talk is mostly about the paper
\begin{center}
M., Sukochev, Lipschitz estimates in quasi-Banach Schatten ideals.
\emph{Math. Ann.} 383 (2022), no.1--2, 571--619.
\end{center}
\end{frame}

\begin{frame}{Plan for this talk}
    \begin{enumerate}
        \item{} Operator Lipschitz functions: some basic concepts and history.
        \item{} Some very light background on Schatten ideals: the problem with $p<1.$
        \item{} Besov spaces and wavelets
    \end{enumerate}
\end{frame}

\begin{frame}{Operator Lipschitz functions}
    Let $H$ be a (complex and separable) Hilbert space, and denote
    the operator norm by $\|\cdot\|_\infty$. A function $f:\Rl\to \Cplx$
    is said to be \emph{operator Lipschitz} if there exists a constant $C_f$ such that
    \begin{equation*}
        \|f(A)-f(B)\|_\infty \leq C_f\|A-B\|_\infty,\quad A,B\in \Bc_{\sa}(H)
    \end{equation*}
    \begin{block}{Question (from Krein)}
        Is every Lipschitz function operator Lipschitz?
        \pause
        That is, does $|f(t)-f(s)|\lesssim |t-s|$ imply that $\|f(A)-f(B)\|_{\infty} \lesssim \|A-B\|_{\infty}?$
    \end{block}
\end{frame}


\begin{frame}{Operator Lipschitz functions}
    \begin{block}{Answer}
        No.\\\pause
        Farforovskaya (1968): There exist Lipschitz functions that are not operator Lipschitz\\\pause
        Kato (1973): The absolute value function $f(t) = |t|$ is not operator Lipschitz\\\pause
        Johnson \& Williams (1975): An operator Lipschitz function is differentiable.
    \end{block}
\end{frame}

\begin{frame}{Finite-dimensional case}
    If $H$ is $N$-dimensional, then
    \[
        \|f(A)-f(B)\|_{\infty} \leq C_{\mathrm{abs}}\log(1+N)\|f\|_{\mathrm{Lip}}\|A-B\|_{\infty}
    \]
    where $C_{\mathrm{abs}}$ is an absolute constant. This is sharp in the order of growth as $N\to\infty.$ I do not know if a sharp estimate for $C_{\mathrm{abs}}$ is known.
\end{frame}


\begin{frame}{Operator Lipschitz functions}
    It is easy to check that sufficiently good functions are operator Lipschitz.\\
    Let's check the function $f(x) = e^{i\xi x}$ for $\xi\in \Rl.$ We have
    \begin{equation*}
        e^{i\xi A}-e^{i\xi B} = i\xi\int_{0}^1 e^{i\xi (1-\theta)A}(A-B)e^{i\xi \theta B}\,d\theta.
    \end{equation*}
    The integral converges in the Bochner sense. The triangle inequality implies
    $$
        \|e^{i\xi A}-e^{i\xi B}\|_{\infty} \leq |\xi|\|A-B\|_{\infty}.
    $$
    \pause
    By Fourier inversion,
    \begin{equation*}
        \|f(A)-f(B)\|_{\infty} \leq \|A-B\|_\infty\cdot 2\pi \|\widehat{\partial f}\|_1.
    \end{equation*}
    \pause
    By Cauchy-Schwarz, $\|\widehat{\partial f}\|_1 \leq \|f'\|_2+\|f''\|_2.$ This is a ``good enough" sufficient condition for most purposes.
\end{frame}

\begin{frame}{Peller's theorem}
    The previous computation was based on Fourier inversion of $f$ and a description of $e^{i\xi A}-e^{i\xi B}$ as an integral (Duhamel's integral).
    \pause
    Using a more subtle description of $e^{i\xi A}-e^{i\xi B}$, and handling the Littlewood-Paley components of $f$
    individually, V. V. Peller has proved the following:
    \begin{theorem}[Peller (1990)]
        If $f$ is Lipschitz and belongs to the homogeneous Besov class $\dot{B}^1_{\infty,1}(\Rl)$ then $f$ is operator Lipschitz.
    \end{theorem}
    \pause
    In other words, if
    \begin{equation*}
        \int_0^\infty \sup_{t\in \Rl} \frac{|f(t-h)-2f(t)+f(t+h)|}{h^2} \,dh + \sup_{t\in \Rl,h> 0} \frac{|f(t+h)-f(t)|}{h} < \infty
    \end{equation*}
    then $f$ is operator Lipschitz. 
    \pause
    For example, if $f' \in W^{1}_{\infty}(\Rl)$ then $f$ is operator Lipschitz.
\end{frame}

\begin{frame}{Peller's operator Bernstein inequality}
    The classical Bernstein inequality states that if $f \in L_{\infty}(\Rl)$ has Fourier transform supported in the interval $[-\sigma,\sigma],$ then
    \[
        \|f\|_{\mathrm{Lip}} \leq C\sigma \|f\|_{\infty}.
    \]\pause
    Peller's theorem is a consequence of his \emph{operator Bernstein inequality}.
    \begin{theorem}[Peller (1990)]
        If $f \in L_{\infty}(\Rl)$ has Fourier transform supported in the interval $[-\sigma,\sigma],$ then
        \[
            \|f\|_{\mathrm{O-Lip}} \leq C\sigma\|f\|_\infty.
        \]
    \end{theorem}
    Here $\|f\|_{\mathrm{O-Lip}}$ is the operator Lipschitz seminorm, i.e.
    \[
        \|f\|_{\mathrm{O-Lip}} := \sup_{A=A^*,B=B^*\in \Bc(H)} \frac{\|f(A)-f(B)\|_{\infty}}{\|A-B\|_{\infty}}.
    \]
\end{frame}

\begin{frame}{Schatten ideals}
    If $T$ is a compact operator on $H$, the singular value sequence of $T$ is defined as
    $$
        \mu(k,T) := \inf\{\|T-R\|_{\infty}\;:\;\mathrm{rank}(R)\leq k\},\quad k\geq 0.
    $$
    (Equivalently, $\mu(T) = \{\mu(k,T)\}_{k=0}^\infty$ is the sequence of eigenvalues of the absolute value $|T|$ arranged in non-increasing order with multiplicities.)\\
    \pause
    Note that $\|T\|_\infty = \mu(0,T) = \|\mu(T)\|_{\ell_\infty}.$
    \pause
    For $1\leq p < \infty$, the Schatten $\Lc_p$-norm of a compact operator $T$ is
    $$
        \|T\|_p := \|\mu(T)\|_{\ell_p} = \left(\sum_{k=0}^\infty \mu(k,T)^p\right)^{\frac1p}.
    $$
    Equivalently, $\|T\|_p = \Tr(|T|^p)^{1/p}.$
    It is not obvious, but this is a norm (i.e. $\|T+S\|_p\leq \|T\|_p+\|S\|_p.$)
\end{frame}

\begin{frame}{$\Lc_p$-operator Lipschitz functions}
    A function $f$ on $\Rl$ is said to be $\Lc_p$-operator Lipschitz if there exists a constant $C_f>0$ such that
    $$
        \|f(A)-f(B)\|_p \leq C_f\|A-B\|_p,\quad A,B \in \Bc_{\sa}(H).
    $$
    By a duality argument, $\Lc_1$-operator Lipschitz is the same thing as operator Lipschitz.\pause
    
    What about $1 < p < \infty$?\pause
    \begin{theorem}[Potapov and Sukochev (2010)]
        For $1 < p < \infty$, all Lipschitz functions are $\Lc_p$-operator Lipschitz.
    \end{theorem}   
    \pause
    For $p=2$ this is almost trivial and has been known for approx.~110 years. For $p\neq 2$, this requires some very deep
    harmonic analysis. \pause
    Last year, Conde-Alonso, Gonz\'alez-P\'erez, Parcet and Tablate have a new proof using operator-valued harmonic analysis.
\end{frame}

\begin{frame}{What about $0 < p < 1$?}
    For $0 < p < 1$, we can still define
    $$
        \|T\|_p := \|\mu(T)\|_{\ell_p} = \Tr(|T|^p)^{\frac1p}.
    $$
    This is not a norm, merely a quasi-norm. There is no triangle inequality, merely a quasi-triangle inequality
    $$
        \|T+S\|_p \leq 2^{\frac11-p}(\|T\|_p+\|S\|_p).
    $$
    \pause
    Nonetheless, we have
    \[
        \|T+S\|_p^p \leq \|T\|_p^p+\|S\|_p^p.
    \]
\end{frame}


\begin{frame}{Geometry in $\Lc_p.$}
    The unit ball $B = \{T\;:\; \|T\|_p\leq 1\}$ in $\Lc_p$ is not convex.

    I.e., if $\xi_1,\ldots,\xi_n\in B$ then it might happen that
    \[
        \theta_1\xi_1+\cdots+\theta_n\xi_n\notin B,\quad |\theta_1|+\cdots+|\theta_n|\leq 1.
    \]
    For this reason the theory of integration $\Lc_p$-valued functions is not straightforward. We could have continuous functions $f \in C([0,1],\Lc_p)$
    whose integral is not in $\Lc_p.$\\
    \pause
    Instead, $B$ is only closed under $p$-convex combinations, i.e.
    \[
        \theta_1\xi_1+\cdots+\theta_n \xi_n \in B,\quad |\theta_1|^p+\cdots+|\theta_n|^p \leq 1.
    \]
\end{frame}


\begin{frame}{$\Lc_p$-Lipschitz functions for $0 < p < 1.$}
    Which functions are Lipschitz in $\Lc_p$ when $0 < p < 1$? \\
    \pause
    At least some functions are, for example $f(t) = (t+\lambda)^{-1}$, $\lambda \in \Cplx\setminus \Rl.$\\
    \pause
    What about $f(t) = \exp(it\xi)$ for $\xi\in \Rl$?
\end{frame} 

\begin{frame}{Periodic functions are not $\Lc_p$-Lipschitz for $0 < p < 1$.}
    A first hint that the $0 < p < 1$ case is interesting comes from the following:
    \begin{lemma}[M. and Sukochev (2022)]
        Let $0 < p < 1$, and let $f$ be a periodic function on $\Rl$. Then $f$ is $\Lc_p$-Lipschitz
        if and only if it is constant.
    \end{lemma}
    What does this imply?\pause
    \begin{itemize}
        \item{} Even $C^\infty$ functions with all derivatives bounded may not be $\Lc_p$-Lipschitz;\\
        \item{} In particular $f(t) = \exp(it\xi)$, $\xi\neq 0$ is not $\Lc_p$-Lipschitz for any $0 < p < 1.$ This means that methods based on a Fourier decomposition
        are unlikely to work.
    \end{itemize}
\end{frame}

\begin{frame}{Fourier multipliers}
    An analogous issue is Fourier multipliers in $L_p(\Circ)$ for $0<p<1.$
    \begin{theorem}
        Let $m \in \ell_{\infty}(\Itgr)$ and $0<p<1.$ The Fourier multiplier
        \[
            T_m:L_2(\Circ)\to L_2(\Circ),\quad T_m(\exp(i\theta n)) = m(n)\exp(i\theta n),\quad n\in \Itgr
        \]
        is bounded on $L_p(\Circ)$ if and only if $m$ has the form
        \[
            m(n) = \sum_{j=0}^\infty c_j \exp(in\zeta_j),\quad n\in \Itgr
        \]
        where $\sum_{j} |c_j|^p < \infty.$
    \end{theorem}
    In other words, the only $L_p(\Circ)$ multipliers for $0<p<1$ are shift operators and $p$-convex combinations of shifts.
\end{frame}

\begin{frame}{Strategies to get $\Lc_p$-operator Lipschitz estimates}
    In the $\Lc_{\infty}$ case, we started with a class of functions $\{\exp(i\xi x)\}_{\xi\in \Rl}$ for which Lipschitz estimates are easy, and derived a more general class by taking convex combinations.

    If we could find some set $\{\psi_j\}$ of functions which we know are $\Lc_p$-Lipschitz, then we could conclude that functions of the form
    \[
        \sum_j c_j\psi_j
    \]
    are also $\Lc_p$-operator Lipschitz.
\end{frame}

\begin{frame}{Strategies to get $\Lc_p$-operator Lipschitz estimates}
    We know that if $f_\lambda(t) = (t+\lambda)^{-1},$ where $\lambda\in \Cplx\setminus \Rl,$ then
    \[
        \|f_{\lambda}\|_{\Lc_p-\mathrm{Lip}} \leq |\Im(\lambda)|^{-2}.
    \]
    Essentially every smooth function on $\Rl$ belongs to the closed convex hull of $\{f_\lambda\}_{\Im(\lambda)\neq 0}.$\\
    \pause
    I tried for a long time to characterise functions $f$ having a decomposition like
    \[
        f(t) = \sum_{j=0}^\infty c_j|\Im(\lambda_j)|^2f_{\lambda_j}(t)
    \]
    where $\sum_{j=0}^\infty |c_j|^p < \infty,$ but with no success.
\end{frame}

\begin{frame}{Wavelet methods}
    It is possible to prove that if $f$ is a compactly supported $C^k$ function where $k > \frac{1}{p}$ then $f$ is $\Lc_p$-Lipschitz.
    \pause
    What is a good way of approximating a general function from compactly supported $C^k$-functions?\pause
    \begin{theorem}[Daubechies (1988)]
        For all $k>0$, there exists a compactly supported $C^k$ function $\psi$ such that the system of translations
        and dilations
        \begin{equation*}
            \psi_{j,k}(t) := 2^{\frac{j}{2}}\psi(2^jt-k),\quad j,k\in \Itgr
        \end{equation*}
        forms an orthonormal basis of $L_2(\Rl).$
    \end{theorem}
\end{frame}


\begin{frame}{A new result}
    Using wavelet methods we can get the following:
    \begin{theorem}[M. and Sukochev (2022)]
        Let $0 < p < 1.$ Let $f \in \dot{B}^{\frac1p}_{\frac{p}{1-p},p}(\Rl)$ be Lipschitz continuous. Then $f$ is $\Lc_p$-Lipschitz and
        $$
            \|f(A)-f(B)\|_p \leq C_{p}(\|f'\|_\infty+\|f\|_{\dot{B}^{\frac1p}_{\frac{p}{1-p},p}(\Rl)})\|A-B\|_p,\quad A,B\in \Bc_{\sa}(H).
        $$
    \end{theorem}
    \pause
    In other words, we require that $f$ be Lipschitz and for some $n>\frac1p$ that
    \begin{equation*}
        \int_0^\infty \left(\int_{-\infty}^\infty \left|\sum_{k=0}^n \binom{n}{k}(-1)^{n-k}f(t+kh)\right|^{\frac{p}{1-p}}\,dt\right)^{1-p} \frac{dh}{h^2} < \infty.
    \end{equation*}
    \pause
    For example, if $f' \in W^{k}_{\frac{p}{1-p}}(\Rl)$ where $k > \frac{1}{p}-1$ then $f$ is $\Lc_p$-Lipschitz.
\end{frame}

\begin{frame}{What else can we do?}
    Wavelets are not new, but their application to this theory is. \pause
    Some other things we can achieve:
    \begin{itemize}
        \item{} For all $n\geq 0$, the inequality
        $$
            \sum_{k=0}^n \mu(k,f(A)-f(B))^p \lesssim (\|f'\|_\infty+\|f\|_{\dot{B}^{\frac{1}{p}}_{\frac{p}{1-p},p}})\sum_{k=0}^n \mu(k,A-B)^p
        $$
        (this recovers the previous result with $n=\infty.$)\pause
        \item{} H\"older-type estimates of the form
        $$
            \|f(A)-f(B)\|_{p} \lesssim_f \||A-B|^{\alpha}\|_p
        $$
        for $f$ in some Besov space.
    \end{itemize}
\end{frame}

\begin{frame}{Wavelet methods}
    Wavelets are analogous to Fourier series, in the sense that if
    \[
        f(t) = \sum_{j\in \Itgr} \sum_{k\in \Itgr} c_{j,k}\psi_{j,k}(t)
    \]
    then the coefficients $c_{j,k}$ for $j>N$ represent oscillations of $f$ on the scale $\sim 2^{-N}.$ A function of the form
    \[
        f(t) = \sum_{j<N} \sum_{k\in \Itgr}c_{j,k}\psi_{j,k}(t)
    \]
    does not oscillate greatly on scales smaller than $2^{-N}.$ This is similar to functions with Fourier transform supported in $[-2^N,2^N].$
\end{frame}

\begin{frame}{Wavelet Bernstein inequality}
    How do we use wavelet methods? The key is again a Bernstein inequality.
    \begin{theorem}[Meyer(?) (1980s)]
        Let $f\in L_{\infty}(\Rl)$ have Wavelet expansion
        \[
            f(t) = \sum_{j\in \Itgr} \sum_{k\in \Itgr} c_{j,k}\psi_{j,k}(t)
        \]
        where $c_{j,k}=0$ for $k>N.$ Then
        \[
            \|f\|_{\mathrm{Lip}} \leq C2^N \|f\|_{\infty}.
        \]
    \end{theorem}
\end{frame}

\begin{frame}{An $\Lc_p$-Lipschitz Bernstein inequality}
        \begin{theorem}[M.-Sukochev (2022)]
        Let $f\in L_{\frac{p}{1-p}}(\Rl)$ have Wavelet expansion
        \[
            f(t) = \sum_{j\in \Itgr} \sum_{k\in \Itgr} c_{j,k}\psi_{j,k}(t)
        \]
        where $c_{j,k}=0$ for $k>N.$ Then
        \[
            \|f\|_{\Lc_p-\mathrm{Lip}} \leq C2^{\frac{N}{p}} \|f\|_{\frac{p}{1-p}}.
        \]
    \end{theorem}
    With $p=1,$ this is the wavelet analogy of Peller's operator Bernstein inequality. For $p<1$ it is new.
\end{frame}

\begin{frame}{Wavelets and Besov spaces}
    It follows from the Wavelet Bernstein inequality that Besov spaces have a very simple characterisation in terms of wavelet coefficients.
    \begin{theorem}[Meyer (1986)]
        Let $s \in \Rl$ and $p,q\in (0,\infty].$ Let $\psi$ be a compactly supported $C^k$ wavelet where $k > -s.$ Then a distribution $f\in \Dc'(\Rl)$
        belongs to the homogeneous Besov space $\dot{B}^{s}_{p,q}(\Rl)$ if and only if
        \begin{equation*}
            \|f\|_{B^s_{p,q}}\approx \sum_{j\in \Itgr} 2^{jq(s+\frac{1}{2}-\frac{1}{p})}\left(\sum_{k\in \Itgr} |\langle f,\psi_{j,k}\rangle|^p\right)^{\frac{q}{p}} < \infty.
        \end{equation*}
    \end{theorem}
    Using the $p$-triangle inequality and the $\Lc_p$-Lipschitz Bernstein inequality, we easily conclude that $\|f\|_{\Lc_p-\mathrm{Lip}} \lesssim \|f\|_{B^{\frac1p}_{\frac{p}{1-p},p}(\Rl)}.$
\end{frame}


\begin{frame}
\structure{\begin{center}
{\Huge{}Thank you for listening!}
\par\end{center}}\end{frame}



\end{document}

